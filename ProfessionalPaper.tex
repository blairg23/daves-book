\documentclass[oneside]{book}
\usepackage{tocbibind}
\usepackage{tocloft}
\usepackage{lipsum}
\usepackage{graphicx} %for including graphics and figures
\usepackage{caption} %for captions outside of floats

%\usepackage{fancyhdr} %%For headers/footers\
%\pagestyle{fancy} %%For fancy headers

\AtBeginDocument{%
  \renewcommand\contentsname{Table of Contents}
}

% Centered title for ToC, LoF, LoT
\renewcommand{\cfttoctitlefont}{\hfill\Huge\bfseries}
\renewcommand{\cftaftertoctitle}{\hfill}
\renewcommand{\cftloftitlefont}{\hfill\Huge\bfseries}
\renewcommand{\cftafterloftitle}{\hfill}
\renewcommand{\cftlottitlefont}{\hfill\Huge\bfseries}
\renewcommand{\cftafterlottitle}{\hfill}

% Leaders for chapter entries
\renewcommand\cftchapdotsep{\cftdotsep}

% Add space to account for new chapter numbering schema
\renewcommand\cftchapnumwidth{3em}
\renewcommand\cftsecindent{3em}

% Redefine representation for chapter (and section) counters
\renewcommand\thechapter{\arabic{chapter}}
\renewcommand\thesection{\arabic{chapter}.\arabic{section}}

%% \fancyhf{}

\begin{document}
%%Title
%\fancyhead[L]{If There's A God, HELP!} %%Header
%\fancyfoot[L]{David A. Allan} %Footer
%\rhead{}
%\renewcommand{\footrulewidth}{1pt}
%\fancyhead[R]{\thepage}
%
%%Fix headers on Table of Contents and List of Figures:
%\fancypagestyle{plain}{
%	\fancyhead{}
%	\fancyfoot{}
%	\fancyhead[L]{If There's a God, HELP!} %%Header
%	\fancyfoot[L]{David A. Allan} %Footer
%	\rhead{}
%	\renewcommand{\footrulewidth}{1pt}
%}

\frontmatter
\title{If There's a God, HELP!\\
	\large Facing Trials with God
}
\author{David A. Allan}
\maketitle
\chapter{IMPORTANT NOTE}
I have donated this book to the church where we serve in Mexico; therefore, I am not charging for it. Instead, I am asking for donations to be sent to LaFuente Riviera at the addresses below. They will provide a tax-deductible receipt for any amount over \$25.00 (however, they will gladly accept any donation amount). For online giving go to Paypal (US). You could also personally give me contributions and I will make sure it gets to LaFuente (make checks out to LaFuente Riviera). Thank you for any help you can give this ministry!\\

\begin{center}
\underline{\textbf{In the United States}}\\
LaFuente Riviera\\
10924 Oro Vista Avenue\\
Sunland, CA. 91040-2025\\
\
\

\underline{\textbf{In Canada}}\\
La Fuente Ministries\\
c/o The Great Commission Foundation\\
P.O. Box 14006\\
Abbotsford, BC, Canada V2T0B4
\
\
Include a note that these funds are for LaFuente Riviera.\\
(If you do not need the tax benefits of Canadian giving, you can use PayPal in the USA for easier giving.)
\end{center}
\textbf{License pending--all material sole property of David A Allan and LaFuente Riviera.}
\chapter{Acknowledgements}
\lipsum[1]
\clearpage
\tableofcontents
\clearpage
\mainmatter
% Chapter 1
\chapter{Out of Control--HELP!}
\

It was July, 1976. I was a twenty-seven year old Canadian young man, born and raised on my parents’ farm in central Alberta, Canada. I was not serving God yet and thought I had a great future ahead of me. I had the keys to my shiny, new four-passenger Cessna 182 in my hand. It had just arrived off Cessna's assembly line in Wichita, Kansas. I had my private pilot's license, which I had acquired three months previously; the adventure of my first real cross country flight was before me. The passengers were my wife, Judy, and her parents, Ivor and Doreen. We were about to fly across Canada to visit my wife's brother, Colin, who was serving in the Canadian Naval base near Halifax, Nova Scotia. The distance was the equivalent of flying from Seattle to New York City (for our American friends); it was approximately 3000 miles, much of which is over rugged terrain, including lakes and water. Looking back on this trip, I would agree with anyone who said, “You’re crazy!” But that is how I lived most of my life, on the edge. Judy was in the right seat reading air maps through all kinds of terrain totally unfamiliar to any of us. (This was long before GPS technology.)

Our first stop was in Regina, Saskatchewan, to fuel the airplane and have lunch. While there, we had to buy some Dramamine for my father-in-law because of his uneasy stomach and nerves. A better choice would have been to turn the aircraft around, head back home, and book airline tickets. Sometimes I get smart too late, so we did carry on. It became apparent that I truly was under-qualified to master such a trip. Part of the trip was through U.S. airspace and customs; it involved landing at unfamiliar airports. 

Three days later we did make it safely to St. John’s, New Brunswick, which is on the Bay of Fundy on the Atlantic Ocean. We stayed overnight and planned to make the final leg of the journey to the Halifax International Airport the following day. The last leg of the journey normally takes about 1 ½ hours in my 140 mph aircraft. We woke up to a partially overcast sky but the weather forecast was acceptable to fly in VFR (visual flight rules). I was not trained or licensed to fly IFR in cloud (instrument flight rules). We were able to get safely across the 50 miles of ocean before entering the province of Nova Scotia, which is basically an island surrounded by ocean. 

Then—guess what? The weather suddenly got much worse than the forecast had predicted. We found ourselves squeezed between mountainous terrain and a low cloud bank. I tried to turn the aircraft around to go back to St. John’s, but soon realized the weather had closed in behind us. I  came upon the Greenwood Air Force Base, which is normally closed to civilian aircraft. In order for me to land there, I needed to declare an emergency. When I talked to the Greenwood Air Traffic Control, I discovered that I would have to go through a great deal of red tape if I declared an emergency. Secondly, the traffic controller told me that the weather had deteriorated dramatically over a very large area. He said there were military aircraft coming in from all over looking for a place to land and some were short on fuel. In light of that, I decided to keep flying and said to myself, “I need to find a road along the ocean on the east side of Nova Scotia and perhaps locate a place to land, or somehow be fortunate enough to find the Halifax Airport.” We were in an impossible jam, flying just above the treetops and below the cloud in unfamiliar territory; that was very dangerous and crazy. I then tried to reverse our course but couldn’t due to the rapidly deteriorating weather, so I decided to climb up into the cloud and hopefully get between layers. 

The reality is that if you haven't been trained to fly in cloud with your instruments, you become disorientated very quickly. In 1-2 minutes there was a different sound in the cockpit and my instruments showed we were spinning out of control. We were probably less than 1000 feet off the ground and my altimeter was recording 1500 ft/min descent with the airspeed near 200 mph. That basically spells “game over” in the next few seconds. \textbf{JUDY CRIED OUT, "IF THERE IS A GOD, HELP!"} In a split second, I saw a clear memory of one of my training exercises back home where I was recovering from a power-on, spiral dive; however, that recovery was in bright sunshine. (Remember, I was not trained for flying in cloud by instrument.) In order to recover from the spiral dive, I instinctively did exactly what I saw myself doing in that flashback memory. By the way, that is not the kind of thing you would normally be inclined to do in a situation like this. I had no time to do anything else; it was blind faith. During the recovery we went through tremendous G-forces; they should have overstressed our airplane or perhaps even torn the wings off. It was a risk I needed to take since I didn’t have enough altitude to recover properly. As the aircraft pulled through the bottom of the dive, we simultaneously came back out of the cloud with the nose of the aircraft level but we were tipped on our side. I was still airborne, although disorientated. Wow, that was close! It looked like you could stretch out your arm and touch the treetops because we were so close to the ground. We were still alive and that was my first miracle. 

The second supernatural intervention followed soon after. We probably had another 90 miles to go; my mind was totally screwed up. I was completely lost and I had no radio contact with the outside world. I began to talk out loud to myself, “Believe the instruments, believe the instruments”. Remarkably I found a paved road which we followed for a few miles. It was quite curvy and I soon realized that there was no place to land. I was low enough to read the road signs but had very little visibility. That’s right, I was low enough to read road signs! Car lights kept appearing out of the light rain and mist. I saw a sign that said “Halifax 17 km” which meant we were approximately 60 kilometers from the airport. The airport was on the other side of the city from my position. How would I ever get across the city flying so low? That was was my last thought. 

All of a sudden the road disappeared, which meant the road went uphill into the mountainside and it was covered in clouds. We were going to die because there was absolutely no way to turn around before crashing into the mountain. We were headed straight into it. As quick as I could blink my eye, I heard a voice through my headset. It was the tower control at the Halifax International Airport. How did we get there? I expected to be dead. The controller was asking, “Which aircraft just showed up on my radar screen on final approach to the runway?” I relayed my airplane identification and asked him, “Can you help me? Where am I?” He said, “You are on the glide slope lined up perfectly one mile from the end of the runway.” In the next minute or so he talked me down onto the runway. The airport had just gone down to minimum visibility for any aircraft to legally land. I taxied to the terminal where I sat on the tarmac for a minute trying to hold back the tears (because big boys don't cry). My legs were numb and I could hardly walk. I realized we had just been supernaturally transported by God 60 km (36 miles) in a split second. If it hadn’t been for God, I wouldn't be telling this story to you.

On our way from the airport to my brother-in-law's home, we had to drive through the most dense fog I had seen in years. Shortly after we arrived there I phoned my Dad back home. He said, “Son, it is so good to hear your voice; are you OK? What were you doing in your airplane a couple of hours ago?” My Dad told me that he was walking across the farmyard some 3000 miles away and had a strong impression to pray for us; he knew we were in grave danger. He went into the house where my Mom was standing at the kitchen sink. She had also felt a sense of urgency and described the feeling as a lightning bolt that hit her. She told my Dad, “We have to pray for Dave and Judy right now!” They went into their bedroom, got down on their knees by the bed, and prayed until they had peace that everything was okay.

I did know that I had felt an incredible, extraordinary presence and power in that airplane. It was way beyond anything I had ever experienced before. That experience captivated my soul for days. As a result, my wife and I had discussions over the next few days about what we were going to do with our lives. We were so grateful to be alive when we didn't deserve to be. Secondly, we agreed that when we got home we would sell the airplane and give up the flying idea. On our return trip we did manage to fly halfway home in good weather but then ran into a weather system that grounded us for a day. It looked like it could be longer than that so Judy and her parents decided to give up on this young, inexperienced pilot and take the train home. I sure couldn't figure that one out, could you? I flew the rest of the trip alone and actually beat them home. On that trip by myself in the airplane, I was completely overwhelmed with questions about what had just happened and also questions about the meaning to life.

When I arrived back in Alberta I thanked one of the flight instructors; he, unknowingly, was responsible for saving my life. Back in April, the week before I was supposed to take the flight test for my pilot’s license, I was booked for my last flying lesson. My regular instructor had been called out on a charter flight; therefore, he was not available to keep his commitment. The flight school couldn't reach me to reschedule so when I showed up at the flying club, another instructor had some time to go flying with me. It was a bonus flight that wasn’t part of my training but he said he would go up with me and do some maneuvers that might be practical in the future. “Sure,” I said, “That would be awesome; let’s go.” We climbed up to 5000 feet on a bright sunny day. He instructed me never to fly in cloud without full instrument training because I could quickly become disoriented and out of control. I wanted him to explain why so he said, “Let me show you. I will blindfold you, ask you to make a couple of thirty degree turns to the right, and then level out. Tell me when you think you are level and I will take off the blindfold.” I followed his instructions and told him when I thought I was level. Wow! to my surprise, when I looked out the windshield, the earth was spinning and coming rapidly towards me. I was in a power-on spiral dive toward the earth and spinning to my right. He said, “Now recover as you have been trained.” By the time I recognized what was happening, reacted to recover, and pulled out of the dive in perfect daylight, we had lost over 1000 feet of altitude. He said, “If you ever get in cloud and you haven't been trained to fly by instruments you will very quickly be out of control.” This also confirmed to me that we should have crashed into the ground back in the cloud at Nova Scotia where we had less than 1000 feet to recover. That recovery during the flight training was the memory that flashed through my mind and saved our lives in July over Nova Scotia. I believe it was not a coincidence, but a God set-up that my regular instructor was called out and another instructor took me up to perform that particular maneuver and recovery back in April. That was something that the flight instructor had never done with any student who was training for their private pilot’s license before me. It is interesting how God goes before us and is there for us even when we are not serving Him. He came to be with us when we were yet sinners that we might be saved (Romans 5:8). 

When I described to my instructor what had happened to us in Nova Scotia, he was amazed and remarked that the chance of recovering like I did in cloud was virtually impossible. The statistics that an inexperienced, untrained pilot would recover and not crash was less than one in a thousand chances. To top it off, that instructor had absolutely no answer for me when I described being transported 36 miles through cloud and across a city to the Halifax Airport in less than a second. In the natural realm, the second miracle was even more remarkable than the first one that day. There is no flight training for being in one place and then suddenly in another with no time between. There are examples in the Bible of suddenly being transported from one place to another; however, it is impossible for man. This was not a figment of my imagination; the passengers with me had no recollection of this lapse in time either. They were just glad to be on the ground safe and sound.

Because of my experience, the local Flight Center put my story in their newsletter and added  some instrument training as part of the requirement for getting an initial pilot’s license. I decided to keep my airplane after all and went on to get my night endorsement and full instrument rating so I was licensed to fly in clouds.

% Chapter 2
\chapter{My Search for Purpose}
\

\section{Flies in the Ointment}
\

% Chapter 3
\chapter{God's Plan or My Plan--Why the Struggle?}\

\section{Putting the Training Wheels Back On}
\

% Chapter 4
\chapter{A Plan for the Future--God Spoke through a Deer!}
\

\section{The Meeting Place, Exodus 33:7-11}
\

\section{You Want My Airplane too--Why Not?}
\

% Chapter 5
\chapter{What's My Place in His Kingdom?}
\

\section{God's Little Lambs}
\

% Chapter 6
\chapter{Keeping the Focus}
\

\section{You Want Me to Fly Again?}
\

% Chapter 7
\chapter{Facing Family Hurts}
\

\section{God's Warning to Pay Attention to Family}
\

\section{Facing an Evan Bigger Family Tragedy}
\

\section{What's Next?}
\

% Chapter 8
\chapter{Job's Trial on Our Farm}
\

\section{Life Leading Into the Trial}
\

\section{A Prophetic Word with a Twist}
\

\section{Job's Trial}
\

\section{Round 1 of My Job Trial}
\

\section{Round 2}
\

\section{Round 3}
\

\section{The Meeting Place Outside the Camp--Exodus 33:7-11}
\

% Chapter 9
\chapter{The Gates of Hell Will Not Prevail}
\

\section{The Call to Ministry--Emotional Healing and Deliverance}
\

% Chapter 10
\chapter{Growing in Authority}
\

% Chapter 11
\chapter{The Development of Encounters Ministry}
\

\section{On Our Way to Rehoboth (Isaac's Third Well)}
\

% Chapter 12
\chapter{Packing our Bags}
\

\section{Reflections}
\

\section{Family Healing at the Farm--Kari}
\

\section{Breaking a Generational Stronghold-Chad}
\

\section{Take Nothing With You}
\

% Chapter 13
\chapter{Becoming More Effective}
\

\section{Encounters in the Streets}
\

% Chapter 14
\chapter{Discovering My True Identity--Battle Over Sonship}
\

\section{My True Identity Encounter in Denver}
\

\section{The Garden and the Gardener--Another Encounter}
\

\section{Meet My Gardener}
\

% Chapter 15
\chapter{Facing the Unexpected}
\

\section{Back to Homeland Canada}
\

\section{God Goes Before Us}
\

\section{Never Say Never}
\

\section{A Dream for Mexico}
\

\section{Putting the Dream on Hold}
\

% Chapter 16
\chapter{Life After Death}
\

\section{Doorway to a New Season}
\

\section{God's Will and God's Timing}
\

\section{Back to the U.S.A.}
\

\section{Settling Into the Flathead Valley, Montana 2011-12}
\

\section{Provision for the Vision}
\

% Chapter 17
\chapter{God said, "This One's on Me"}
\

\section*{Round 1}
\

\lipsum[1]
\section{Round 2}
\

\lipsum[1]
\section{Round 3}
\

\section{Concluding Remarks}
\


\backmatter
\clearpage

\chapter*{Some Testimonies from the Encounters Ministry}
\chapter*{APPENDIX}
\addtocontents{toc}{\protect\addvspace{10pt}}
\addtocontents{toc}{\textbf{APPENDIX}}

\end{document}